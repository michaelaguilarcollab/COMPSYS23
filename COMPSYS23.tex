\documentclass[conference]{IEEEtran}
\IEEEoverridecommandlockouts
% The preceding line is only needed to identify funding in the first footnote. If that is unneeded, please comment it out.
\usepackage{cite}
\usepackage{amsmath,amssymb,amsfonts}
\usepackage{algorithmic}
\usepackage{graphicx}
\usepackage{textcomp}
\usepackage{xcolor}
\usepackage{hyperref}

\def\BibTeX{{\rm B\kern-.05em{\sc i\kern-.025em b}\kern-.08em
    T\kern-.1667em\lower.7ex\hbox{E}\kern-.125emX}}
\begin{document}

\title{Centralized Composable HPC Management with the OpenFabrics Managment Framework\thanks{Sandia National Laboratories is a multi-mission laboratory managed and operated by National Technology \& Engineering Solutions of Sandia, LLC, a wholly owned subsidiary of Honeywell International Inc., for the U.S. Department of Energy’s National Nuclear Security Administration under contract DE-NA0003525.}\thanks{The OpenFabrics Alliance (OFA) mission is to accellerate the development and adoption of advanced fabrics for the benefit of the advanced networks ecosystem, which is accomplished by: creating opportunities for collaboration among those who develop and deploy such fabrics, incubating and evolving vendor independent open source software for fabrics, and supporting and promoting the use of such fabric technology software.}
}

\author{\IEEEauthorblockN{1\textsuperscript{st} Michael Aguilar}
\IEEEauthorblockA{\textit{High Performance Computing Systems} \\
\textit{Sandia National Laboratories}\\
Albuquerque, New Mexico, USA \\
mjaguil@sandia.gov}
\and
\IEEEauthorblockN{2\textsuperscript{nd} Phil Cayton}
\IEEEauthorblockA{\textit{Senior Staff Engineer} \\
\textit{Intel Corporation}\\
Hillsboro, Oregon, USA \\
phil.cayton@intel.com}
\and
\IEEEauthorblockN{3\textsuperscript{rd} Michele Gazetti}
\IEEEauthorblockA{\textit{Research Engineer} \\
\textit{IBM Research Europe}\\
Dublin, Ireland\\
Michele.Gazzetti1@ibm.com}
\and
\IEEEauthorblockN{3\textsuperscript{rd} Russ Herrell }
\IEEEauthorblockA{\textit{EE system architect at Hewlett-Packar} \\
\textit{Hewlett Packard Enterprise}\\
Fort Collins, CO, USA\\
rherrell@hpe.com}
}

\maketitle

\begin{abstract}
Current HPC systems are limited in performance and architecture by the the stranding of disaggregated components. In current HPC systems, common tools do not exist to manage increasingly diverse network fabrics and component resources, such as, CPUs, hierarchical memory components, remote NVMe, and accelerators.  Due to lack of access to good Resource Management tools, current HPC systems suffer from inefficient over-design where resources must be provided locally, often in node, to make In instances that HPC running jobs could potentially require the resources.  In addition, the inability to dynamically share available resources, as they are required by running batch jobs can lead to reduced and inefficient computational performance and running job failure.  Centralized resource management can potentially mitigate, out-of-memory conditions, IO thashing, stranding of available resources, such as, CPUs, GPUs, and memories, and provide dynamic network fail-over.  It is clear that Resource Management, using a standardized interface, can enable clients to monitor, compose, and intelligently provision resources, in beneficial ways.

The OpenFabrics Management Framework (OFMF) is a open-source Resource Manager being developed by the OpenFabric Alliance (OFA), the DMTF, SNIA, and the CXL Consortium. The OFMF implements Redfish and Swordfish through the implementation of a Swordfish Endpoint Emulator and network Agents.  The Emulator provides centralized resource monitoring and command control for the attached hardware resources and network fabrics, through matching network Agents. A Composability Manager, integrated with the OFMF, can mitigate stranded resources by providing a method for sharing hardware, CPUs, GPUs, NVM, and memories.  Integration with the OFMF provides the capability for dynamic provisioning of resources to maintain running client computations.  

The OFMF is designed for configuring fabric interconnects and managing composable disaggregated resources in dynamic HPC infrastructures using client-friendly abstractions.  The goal of the OFMF is to enable interoperability through common interfaces to enable client Managers to efficiently connect workloads with resources in a complex heterogenous ecosystem without having to worry about the underlying network technology.  

In addition, the OFMF project scope includes adding functional components to manage e.g., Network, GPU, and CPU Composition, Fabric Attached Memory, Fabric Attached Storage, Platform Composition, and Monitoring. 
\end{abstract}

\begin{IEEEkeywords}
component, formatting, style, styling, insert
\end{IEEEkeywords}

\section{Introduction}

Traditional HPC compute clusters are created by combining separate compute servers over network fabric devices to form the cluster.  Each individual compute server is statically provisioned with its own CPUs, memory devices, accelerator cards, and storage devices to incorporate as many different application runtime requirements as possible\cite{beowulf}. This need to incorporate 'all of the options' often results in resource overprovisioning, makes traditional HPC architectures less flexible and more inefficient, and can lead to situations where application jobs are more prone to run-time failure. This paper target HPC infrastructure, however it is worth to note that resources overprovisioning and inefficient use of hardware are common issues to any large scale computing facility \cite{borg-google, pond}.

For example, design considerations that lead to an under-estimation of compute server memory resources can cause out-of-memory conditions.  In another example, IO server memory oversubscription can result in filesystem failure can occur due to virtual memory page swap thrashing, and eventually application failure when the dynamic addition of memory would be able to help mitigate this problem.  

Another issue with the architectural inflexibility of current, siloed, HPC architecture is that it frequently results in overprovisioned or stranded resources. Stranded resources are those that are either are on a compute server that, due to a lack of other resources (e.g., CPU), is unavailable to a workload, or that have been assigned to a workload that isn't making use of them and are unavailable to be used by other workloads. Overprovisioned resources are those that are either underused, or unused and idle for the current workloads but still draw energy and cooling. Energy wasted in data centers is becoming an increasingly important issue\cite{eere}.

The facility costs of large scale HPC systems including cooling and energy usage is becoming more of an issue. Today, 4\% of the energy produced in the world is used in data centers, up from 2\% of energy usage used in data centers 2 years ago\cite{dw,vmware}. 

A solution to addressing the overprovisioning and computational efficiency limitations, as well as hardware and operating costs, of integrated, siloed, systems is the use of Composable Disaggregated Infrastructures (Figure \ref{fig:stranded}).

\begin{figure}
\centerline{\includegraphics[width=\columnwidth]{Slide3.jpeg}}
\caption{More Efficiency in Composable HPC Use of Resources.} 
\label{fig:stranded}
\end{figure}

With Composable Disaggregated Infrastructures, computational resources are not statically provisioned in servers, but instead are physically disaggregated and connected through high-speed/low-latency network fabrics.  These resources can be dynamically provisioned and re-provisioned to client applications, as needed and are thus not only more efficient to manage by removing unnecessary hardware, but help reduce energy consumption and datacenter cooling costs.In this type of architecture, resources are grouped in shared 'pools' that are accessed across high-speed, low latency fabrics (Figure \ref{fig:Pools}). 


\begin{figure}
\centerline{\includegraphics[width=\columnwidth]{pools2.jpeg}}
\caption{Localized Disaggregated Resource Pools Connected by Fabrics.} 
\label{fig:Pools}
\end{figure}
  
Network disaggregation is already common for storage devices (e.g., NVMe-oF); current trends are pushing this paradigm further, extending it to computational engines, memory elements, accelerators and eventually to all forms of compute resources required by modern HPC applications. However, disaggregated resource types are increasingly being accessed over a variety of fabric types and technologies; and being able to fully orchestrate these resources in a dynamic, heterogenous environment requires managing those fabrics and the hardware resources that may be accessed thereon. The management and optimization of such a diverse set of fabrics and fabric technologies to realize the benefits of Composable Disaggregated Infrastructures is quickly becoming a complex issue to solve for infrastructure managers, especially in heterogenous multi-vendor environments, with multiple vendor-sourced hardware and the ever-expanding collection of proprietary APIs and tools. 

Currently, there is no common open-source manager interface or model available to configure the resource pools and the fabrics that link them with applications. So, every tool and every middleware library provider needs unique calls to specific fabric managements stack for each available fabric. An HPC cluster can end up with very diverse administration domains with administrators having to manage each fabric differently through different tools.

The industry needs interoperatbility through common interfaces to enable Composability Managers to efficiently connect workloads with resources in a dynamic ecosystem while being able to the abstractly control the underlying network interconnect. This paper describes the \textit{OpenFabric Management Framework} (OFMF). The OFMF is an open-source API, tool set, and central repository being designed and developed for centralized management of composable resources over dissimilar fabrics, and for manipulation of connected resources using client-friendly abstractions. The OFMF provides a framework that makes it possible to dynamically configure fabric interconects to pair client workloads.

  



\section{Related Work}
As HPC systems have grown, their associated storage systems have been widely identified as a potential bottleneck. One major concern was the viability of the checkpoint-restart pattern of resilience, as compute and memory progress was outpacing parallel file system performance. However, the increasing prominence of data analytics workloads on HPC system presented further risk to the viability of upcoming system architectures, yielding an increased emphasis on the search for alternative solutions.

One early solution was to create technologies specifically to increase the performance of checkpointing, even at the expense of later restarts. Solutions in this space included PLFS~\cite{plfs}, Zest~\cite{zest}, and buddy-checkpointing via SCR~\cite{scr}. These techniques were often regarded as stop-gap solutions, as they decreased checkpoint reliance (or checkpoint requirements) on parallel file systems. However, these techniques did not address the core difficulty of parallel file systems struggling to meet new performance requirements.

The next step was development of new types of storage systems to allow storage workloads to largely avoid parallel file system use. A new class of technology termed ``burst buffers'' was developed~\cite{burstbuffers}. Burst buffers are typically flash-based storage appliances distributed throughout an HPC system, designed to service workloads in a more local fashion. Space within them are often allocated via the job scheduler, allowing some level of performance isolation from other jobs using other burst buffer components in remote parts of the system. Although sometimes still called burst buffers, this term has fallen out of favor because they are now more widely used than the original application, absorbing bursts of I/O generated by checkpoint restart. Instead, these systems (like DataWarp and DDN Infinite Memory Engine) are also used extensively for data analytics, helping resolve the original shared parallel file system contention problem.

A relatively new development is the on-demand parallel file system. BeeGFS-On-Demand~\cite{beeond} is a commonly available instance of this technology. Lustre On Demand~\cite{lustre-on-demand} is a forthcoming design based on this concept. The main idea is to assemble node-local resources (like on-node SSDs and NVMe) to create a parallel file system independent of the centralized file system. This is accomplished by launching file system daemons within each node to serve requests from the clients. One common deployment allows for individual HPC jobs to assemble their own parallel file systems, eliminating parallel file system contention entirely.

While managing parallel file system contention has been a widely discussed topic~\cite{managing-contention}, the impact of I/O processes on compute-bound tasks has not yet been deeply explored. Microkernel research from past decades has highlighted the impact of daemons commonly found in Linux on very large scale, tightly coupled jobs~\cite{daemon-interference}. 

\section{Implementation}

\subsection{Design Considerations for a Composability Manager}

The larger the HPC system, the greater the potential impact of dynamic composability of disaggregated components to energy efficiency and computational stability.  However, a centralized composabilty management layer must be scalable to be able to handle massive amounts of hardware telemetry, device states, device capabilities, and subscription information from large numbers of resources.  The OFMF is a centralized abstract management layer that exposes a RESTful API \cite{restful} and incorporates DMTF Redfish \cite{redfish} and SNIA Swordfish \cite{swordfish} schemas to enable infrastructure clients (e.g., users, management software, programming frameworks) to manage composition of, and fabric configuration to computational resources.  The OFMF transactions are stateless and lightweight, consisting of JSON data carried on Open Data Format (OData).  The Component Management Agents and the Composability Layer interact with the OFMF layer.  Through the use of Kafka \cite{kafka} or RabbitMQ \cite{rabbitmq} interfaces, the OFMF is designed to be able to scale.

Figure \ref{fig:ofmf} shows a higher level architectural diagram of the components making up the OFMF and a composability solution.  The left side of the diagram shows the user, admininisration, orchestration, automation, etc. Clients can be various Workload Managers, application and run-time libraries, monitoring systems, and System Administrators.  Clients interact with the Composability Layer between clients and the OFMF. The Composability Layer manages hardware resources to best provide run-time computational performance, energy efficiency, and resource monitoring by applying policies and updating subscribed clients with events. The Composability Layer allows clients to track the current state and coordinate resources that are within a disaggregated HPC system.

\begin{figure}[ComposableSolution]
  \centerline{\includegraphics[width=\columnwidth]{ComposabilityHL_Diagram.jpeg}}
  \caption{Composability Solution based upon the OpenFabrics Management Framework}
  \label{fig:ofmf}
\end{figure}

The middle section of the diagram is a representation of the functional blocks making up the OFMF.  In the OFMF, an HPC's disaggregated infrastructure is represented under a single Redfish tree that includes all the fabrics and computational resources available. The OFMF services are subscription-based and represent a central repository for telemetry information, provisioning, and event logs.  Client requests received to the OFMF through the Composability Layer are forwarded to the appropriate fabric manager via dedicated light-weight technology-specific Agents. 

The Agents on the right translate between the OFMF and network fabric-specific providers.  These Agents provide access to network fabrics and trigger them to make the actual changes to their resources in their own technology-specific manner with their own technology-specific configuration tools.  

\subsection{Dynamic Composability In Action}




\section{Conclusions and Future Work}

In this paper, we reviewed a set of experiments where we combined IOR, a storage benchmark, against HPL, a compute benchmark simulating a tightly coupled linear algebra application. We ran these benchmarks on a system with per-allocation BeeOND file system capabilities, which gives a user a private file system by running storage services on compute nodes alongside running applications. Our goal was to show whether heavy storage use could impact compute task performance.

We were able to show that, even for large compute jobs (128 nodes, 7168 cores) paired with a small number of I/O processes (1 node, 56 processes), there was significant impact to the HPL runtimes. In some cases, HPL runtime was increased by up to 52\%. Surprisingly, we were also able to measure a potential impact (up to 2.5\% on 64 nodes) from simply running BeeOND processes alongside HPL without any I/O traffic. Such a phenomenon indicates that running BeeOND daemons on dedicated cores is a reasonable strategy as job node counts increase.

This research has uncovered a large number of future research directions. While we have uncovered CPU-based performance impacts from I/O daemons, we did not receive clear results regarding impact of metadata services. This is likely because the IOR benchmark parameters we used did not heavily exercise the metadata facilities of BeeOND. We intend to design new experiments that will specifically exercise metadata operations. We also intend to explore user-controlled placement of BeeOND processes to mitigate compute performance impact.


\bibliographystyle{plain}
%\bibliography{COMPSYS23}

\begin{thebibliography}{20}


\bibitem{beowulf}
  Beowulf Cluster,
  url = "https://en.wikipedia.org/wiki/Beowulf_cluster",
  note = "[Online: Accessed: Feb 1, 2023]".

\bibitem{eere}
  Data Centers and Servers - Buildings,
  url = "https://www.energy.gov/eere/buildings/data-centers-and-servers",
  2021.
  
\bibitem{dw}
  Deutsche Welle,
  Data centers keep energy use steady despite big growth,
  url="https://www.dw.com/en/data-centers-energy-consumption-steady-despite-big-growth-because-of-increasing-efficiency/a-60444548",
  January 24, 2022.
  
\bibitem{vmware}
  WMware,
  VMware : Why Energy Sobriety Should Be Top of Mind for Every Business Leader This Year,
  url="https://www.marketscreener.com/quote/stock/VMWARE-INC-58476/news/VMware-Why-Energy-Sobriety-Should-Be-Top-of-Mind-for-Every-Business-Leader-This-Year-42823597/"
  01/26/2023
  
\bibitem{fuzzball}
  CtrlIQ Fuzzball,
  url="https://ciq.co/products/fuzzball/hpc/",
  2023.
  
\bibitem{flux}
  Flux:  A Fully Hierarchical Workload Manager for Supercomputing,
  url="https://ipo.llnl.gov/sites/default/files/2022-02/Flux_RD100_Final.pdf",
  2023. 
  
\bibitem{rlinker}
  Composable Infrastructure Market by Type, Vertical And Region - Global Forecast to 2023,
  url="https://www.reportlinker.com/p05620908/Composable-Infrastructure-Market-by-Type-Vertical-And-Region-Global-Forecast-to.html",
  2018.
  
\bibitem{liqid}
  Liqid touts composable infrastructure at Dell Technologies World,
  url="https://venturebeat.com/data-infrastructure/liqid-touts-composable-infrastructure-at-dell-technologies-world/",
  May 3, 2022.
  
\bibitem{gigaio}
  url="https://gigaio.com/",
  note = "[Online: Accessed: Feb 1, 2023]". 
  
\bibitem{liqidmf}
  Liqid introduces multi-fabric support for composable infrastructure,
  url = "https://www.itopstimes.com/itops/liqid-introduces-multi-fabric-support-for-composable-infrastructure/",
  April 29th, 2019.
  
\bibitem{cdi}
  Composable disaggregated infrastructure,
  url = "https://en.wikipedia.org/wiki/Composable_disaggregated_infrastructure",
  note = "[Online: Accessed: Feb 1, 2023]".

\bibitem{whatcdi}
  What is composable infrastructure?,
  url="https://www.networkworld.com/article/3266106/what-is-composable-infrastructure.html",
  MAR 27, 2018.

\bibitem{fungible}
  Fungible, Inc.,
  url = "https://en.wikipedia.org/wiki/Fungible_Inc.",
  note = "[Online: Accessed: Feb 1, 2023]".


\end{thebibliography}
\end{document}
