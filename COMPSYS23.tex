\documentclass[conference]{IEEEtran}
\IEEEoverridecommandlockouts
% The preceding line is only needed to identify funding in the first footnote. If that is unneeded, please comment it out.
\usepackage{cite}
\usepackage{amsmath,amssymb,amsfonts}
\usepackage{algorithmic}
\usepackage{graphicx}
\usepackage{textcomp}
\usepackage{xcolor}
\usepackage{hyperref}

\def\BibTeX{{\rm B\kern-.05em{\sc i\kern-.025em b}\kern-.08em
    T\kern-.1667em\lower.7ex\hbox{E}\kern-.125emX}}
\begin{document}

\title{Evaluating On-Demand Parallel File System Impacts on Compute-Bound Tasks\thanks{Sandia National Laboratories is a multimission laboratory managed and operated by National Technology \& Engineering Solutions of Sandia, LLC, a wholly owned subsidiary of Honeywell International Inc., for the U.S. Department of Energy’s National Nuclear Security Administration under contract DE-NA0003525.}
}

\author{\IEEEauthorblockN{1\textsuperscript{st} Matthew L. Curry}
\IEEEauthorblockA{\textit{Center for Computing Research} \\
\textit{Sandia National Laboratories}\\
Albuquerque, New Mexico, USA \\
mlcurry@sandia.gov}
\and
\IEEEauthorblockN{2\textsuperscript{nd} Michael Aguilar}
\IEEEauthorblockA{\textit{High Performance Computing Systems} \\
\textit{Sandia National Laboratories}\\
Albuquerque, New Mexico, USA \\
mjaguil@sandia.gov}
\and
\IEEEauthorblockN{3\textsuperscript{rd} Shyamali Mukherjee}
\IEEEauthorblockA{\textit{Computation and Analysis for National Security} \\
\textit{Sandia National Laboratories}\\
Livermore, California, USA \\
smukher@sandia.gov}
}

\maketitle

\begin{abstract}
Current HPC systems are limited in performance and architecture by the the stranding of disaggregated components. In current HPC systems, common tools do not exist to manage both network fabrics and component resources, such as, CPUs, hierarchical memory components, and accelerators.  Due to lack of access to good resource management tools, current HPC systems suffer from inefficient over-design where resources must be provided locally, often in node, to make In instances that HPC running jobs could potentially require the resources.  In addition, the inability to dynamically share available resources, as they are required by running batch jobs can lead to reduced and inefficient computational performance and running job failure.  Centralized resource management can potentially mitigate, Out-of-Memory conditions, IO thashing, stranding of available resources, such as, CPUs, GPUs, and memories, and provide dynamic network fail-over.  It is clear that resource management, using a standardized interface, can enable clients to monitor, compose, and intelligently provision resources, in beneficial ways.

The OpenFabrics Management Framework (OFMF) is a open sourcec resource manager being developed by the OpenFabric Alliance, the DMTF, SNIA, and the CXL Consortium. The OFMF implements Redfish and Swordfish through the implementation of a Swordfish Endpoint Emulator and network Agents.  The Emulator provides centralized resource monitoring and command control for the attached hardware resources and network fabrics, through matching network Agents. A Composability Manager, integrated with the OFMF, can mitigate stranded resources by providing a method for sharing hardware, CPUs, GPUs, and memories.  Integration with the OFMF provides the capability of dynamic provisioning of resources to maintain running client computations.
\end{abstract}

\begin{IEEEkeywords}
component, formatting, style, styling, insert
\end{IEEEkeywords}

\section{Introduction}

Traditional HPC compute clusters are created by combining separate compute servers over network fabric devices to form the cluster.  Each individual compute server is statically provisioned with its own CPUs, memory devices, accelerator cards, and storage devices to incorporate as many different application runtime requirements as possible\cite{beowulf}. This need to incorporate 'all of the options' often results in resource overprovisioning, makes traditional HPC architectures less flexible and more inefficient, and can lead to situations where application jobs are more prone to run-time failure. This paper target HPC infrastructure, however it is worth to note that resources overprovisioning and inefficient use of hardware are common issues to any large scale computing facility \cite{borg-google, pond}.

For example, design considerations that lead to an under-estimation of compute server memory resources can cause out-of-memory conditions.  In another example, IO server memory oversubscription can result in filesystem failure can occur due to virtual memory page swap thrashing, and eventually application failure when the dynamic addition of memory would be able to help mitigate this problem.  

Another issue with the architectural inflexibility of current, siloed, HPC architecture is that it frequently results in overprovisioned or stranded resources. Stranded resources are those that are either are on a compute server that, due to a lack of other resources (e.g., CPU), is unavailable to a workload, or that have been assigned to a workload that isn't making use of them and are unavailable to be used by other workloads. Overprovisioned resources are those that are either underused, or unused and idle for the current workloads but still draw energy and cooling. Energy wasted in data centers is becoming an increasingly important issue\cite{eere}.

The facility costs of large scale HPC systems including cooling and energy usage is becoming more of an issue. Today, 4\% of the energy produced in the world is used in data centers, up from 2\% of energy usage used in data centers 2 years ago\cite{dw,vmware}. 

A solution to addressing the overprovisioning and computational efficiency limitations, as well as hardware and operating costs, of integrated, siloed, systems is the use of Composable Disaggregated Infrastructures (Figure \ref{fig:stranded}).

\begin{figure}
\centerline{\includegraphics[width=\columnwidth]{Slide3.jpeg}}
\caption{More Efficiency in Composable HPC Use of Resources.} 
\label{fig:stranded}
\end{figure}

With Composable Disaggregated Infrastructures, computational resources are not statically provisioned in servers, but instead are physically disaggregated and connected through high-speed/low-latency network fabrics.  These resources can be dynamically provisioned and re-provisioned to client applications, as needed and are thus not only more efficient to manage by removing unnecessary hardware, but help reduce energy consumption and datacenter cooling costs.In this type of architecture, resources are grouped in shared 'pools' that are accessed across high-speed, low latency fabrics (Figure \ref{fig:Pools}). 


\begin{figure}
\centerline{\includegraphics[width=\columnwidth]{pools2.jpeg}}
\caption{Localized Disaggregated Resource Pools Connected by Fabrics.} 
\label{fig:Pools}
\end{figure}
  
Network disaggregation is already common for storage devices (e.g., NVMe-oF); current trends are pushing this paradigm further, extending it to computational engines, memory elements, accelerators and eventually to all forms of compute resources required by modern HPC applications. However, disaggregated resource types are increasingly being accessed over a variety of fabric types and technologies; and being able to fully orchestrate these resources in a dynamic, heterogenous environment requires managing those fabrics and the hardware resources that may be accessed thereon. The management and optimization of such a diverse set of fabrics and fabric technologies to realize the benefits of Composable Disaggregated Infrastructures is quickly becoming a complex issue to solve for infrastructure managers, especially in heterogenous multi-vendor environments, with multiple vendor-sourced hardware and the ever-expanding collection of proprietary APIs and tools. 

Currently, there is no common open-source manager interface or model available to configure the resource pools and the fabrics that link them with applications. So, every tool and every middleware library provider needs unique calls to specific fabric managements stack for each available fabric. An HPC cluster can end up with very diverse administration domains with administrators having to manage each fabric differently through different tools.

The industry needs interoperatbility through common interfaces to enable Composability Managers to efficiently connect workloads with resources in a dynamic ecosystem while being able to the abstractly control the underlying network interconnect. This paper describes the \textit{OpenFabric Management Framework} (OFMF). The OFMF is an open-source API, tool set, and central repository being designed and developed for centralized management of composable resources over dissimilar fabrics, and for manipulation of connected resources using client-friendly abstractions. The OFMF provides a framework that makes it possible to dynamically configure fabric interconects to pair client workloads.

  



\section{Related Work}
As HPC systems have grown, their associated storage systems have been widely identified as a potential bottleneck. One major concern was the viability of the checkpoint-restart pattern of resilience, as compute and memory progress was outpacing parallel file system performance. However, the increasing prominence of data analytics workloads on HPC system presented further risk to the viability of upcoming system architectures, yielding an increased emphasis on the search for alternative solutions.

One early solution was to create technologies specifically to increase the performance of checkpointing, even at the expense of later restarts. Solutions in this space included PLFS~\cite{plfs}, Zest~\cite{zest}, and buddy-checkpointing via SCR~\cite{scr}. These techniques were often regarded as stop-gap solutions, as they decreased checkpoint reliance (or checkpoint requirements) on parallel file systems. However, these techniques did not address the core difficulty of parallel file systems struggling to meet new performance requirements.

The next step was development of new types of storage systems to allow storage workloads to largely avoid parallel file system use. A new class of technology termed ``burst buffers'' was developed~\cite{burstbuffers}. Burst buffers are typically flash-based storage appliances distributed throughout an HPC system, designed to service workloads in a more local fashion. Space within them are often allocated via the job scheduler, allowing some level of performance isolation from other jobs using other burst buffer components in remote parts of the system. Although sometimes still called burst buffers, this term has fallen out of favor because they are now more widely used than the original application, absorbing bursts of I/O generated by checkpoint restart. Instead, these systems (like DataWarp and DDN Infinite Memory Engine) are also used extensively for data analytics, helping resolve the original shared parallel file system contention problem.

A relatively new development is the on-demand parallel file system. BeeGFS-On-Demand~\cite{beeond} is a commonly available instance of this technology. Lustre On Demand~\cite{lustre-on-demand} is a forthcoming design based on this concept. The main idea is to assemble node-local resources (like on-node SSDs and NVMe) to create a parallel file system independent of the centralized file system. This is accomplished by launching file system daemons within each node to serve requests from the clients. One common deployment allows for individual HPC jobs to assemble their own parallel file systems, eliminating parallel file system contention entirely.

While managing parallel file system contention has been a widely discussed topic~\cite{managing-contention}, the impact of I/O processes on compute-bound tasks has not yet been deeply explored. Microkernel research from past decades has highlighted the impact of daemons commonly found in Linux on very large scale, tightly coupled jobs~\cite{daemon-interference}. 

\section{Implementation}

\subsection{Design Considerations for a Composability Manager}

The larger the HPC system, the greater the potential impact of dynamic composability of disaggregated components to energy efficiency and computational stability.  However, a centralized composabilty management layer must be scalable to be able to handle massive amounts of hardware telemetry, device states, device capabilities, and subscription information from large numbers of resources.  The OFMF is a centralized abstract management layer that exposes a RESTful API \cite{restful} and incorporates DMTF Redfish \cite{redfish} and SNIA Swordfish \cite{swordfish} schemas to enable infrastructure clients (e.g., users, management software, programming frameworks) to manage composition of, and fabric configuration to computational resources.  The OFMF transactions are stateless and lightweight, consisting of JSON data carried on Open Data Format (OData).  The Component Management Agents and the Composability Layer interact with the OFMF layer.  Through the use of Kafka \cite{kafka} or RabbitMQ \cite{rabbitmq} interfaces, the OFMF is designed to be able to scale.

Figure \ref{fig:ofmf} shows a higher level architectural diagram of the components making up the OFMF and a composability solution.  The left side of the diagram shows the user, admininisration, orchestration, automation, etc. Clients can be various Workload Managers, application and run-time libraries, monitoring systems, and System Administrators.  Clients interact with the Composability Layer between clients and the OFMF. The Composability Layer manages hardware resources to best provide run-time computational performance, energy efficiency, and resource monitoring by applying policies and updating subscribed clients with events. The Composability Layer allows clients to track the current state and coordinate resources that are within a disaggregated HPC system.

\begin{figure}[ComposableSolution]
  \centerline{\includegraphics[width=\columnwidth]{ComposabilityHL_Diagram.jpeg}}
  \caption{Composability Solution based upon the OpenFabrics Management Framework}
  \label{fig:ofmf}
\end{figure}

The middle section of the diagram is a representation of the functional blocks making up the OFMF.  In the OFMF, an HPC's disaggregated infrastructure is represented under a single Redfish tree that includes all the fabrics and computational resources available. The OFMF services are subscription-based and represent a central repository for telemetry information, provisioning, and event logs.  Client requests received to the OFMF through the Composability Layer are forwarded to the appropriate fabric manager via dedicated light-weight technology-specific Agents. 

The Agents on the right translate between the OFMF and network fabric-specific providers.  These Agents provide access to network fabrics and trigger them to make the actual changes to their resources in their own technology-specific manner with their own technology-specific configuration tools.  

\subsection{Dynamic Composability In Action}




\subsection{Experimental Procedure}

For this study, an experiment is a multi-node HPL task run in the same compute allocation with an IOR task of various sizes. These tasks are placed on non-overlapping sets of nodes, creating a situation without any explicit CPU contention. By measuring the HPL completion time of each configuration, we will be able to detect significant runtime impacts.

Figure~\ref{fig:process-layout} demonstrates how tasks are laid out in an allocation. From this basic structure, we created a set of five experiment classes to measure different phenomenon.
\begin{itemize}
\item {\bf HPL-Only:} $k=0,m=0$. This is the control experiment that excludes IOR as a factor. Since this shared experimental infrastructure, BeeOND daemons were configured and started, but the only task running in the compute allocation is an $n$-node HPL job.
\item {\bf Matching Lustre:} $k=0,m=n$. This can be considered another type of control experiment that excludes BeeOND as a factor, while still allowing IOR to potentially demonstrate an impact. This is valuable to determine whether the file system traffic from IOR can perturb an HPL running on other allocated nodes. Crucially, this is the only configuration that does not load any BeeOND daemons.
\item {\bf Single BeeOND:} $k=0,m=1$. This demonstrates the impact of including a single node running a data-intensive workload. 
\item {\bf Matching BeeOND:} $k=0,m=n$. This demonstrates the impact of a larger number of data-intensive processes on HPL performance.
\item {\bf Matching BeeOND (no meta):} $k=1,m=n$. This demonstrates the same type of potential impact as the Matching BeeOND test, but explicitly places a separate task on the same node as the metadata server. This allows us to ensure that the multi-node HPL is not colocated with the metadata server or other management servers we place alongside it. Therefore, the multi-node HPL will only experience performance impacts from the object storage server.
\end{itemize}

\begin{figure}[htbp]
\centerline{\includegraphics[width=\columnwidth]{process-layout}}
\caption{An illustration demonstrating possible configurations of the experiment.}
\label{fig:process-layout}
\end{figure}

\subsubsection{HPL}
We used HPL~\cite{hpl} as a well-defined compute task to measure. We specified the sizes by starting from a well-performing single-node specification that uses most of the memory on a single node (128MB). When run alone, this takes less than 15 minutes to complete. We then extrapolated to higher node counts by approximating the same amount of work, thus approximately preserving the total runtime of each experiment. This means that, while runs of the same node count are comparable, runs of different node counts are not directly comparable. The problem sizes we used are in Table~\ref{tab:hpl-params}. %We were reminded that Q should be larger than P (\url{https://icl.utk.edu/hpcc/faq/index.html#124}), but we do not believe that this causes any issues with our measurements.


\begin{table}[htbp]
\caption{HPL Parameters by Node Count}
\begin{center}
\begin{tabular}{|c|c|c|c|}
\hline
\textbf{Node Count}&\textbf{Row Count (N)} & \textbf{Grid P} & \textbf{Grid Q} \\
\hline
1 & 91048 & 7 & 8 \\
2 & 114713 & 14 & 8 \\
4 & 144529 & 14 & 16 \\
8 & 182096 & 28 & 16 \\
16 & 229427 & 28 & 32 \\
32 & 289059 & 56 & 32 \\
64 & 364192 & 56 & 64 \\
128 & 458853 & 112 & 64 \\
%256 & 578119 & 112 & 128 \\
\hline
\end{tabular}
\label{tab:hpl-params}
\end{center}
\end{table}

\subsubsection{IOR}
We used IOR to approximate a data-intensive write task. We designed the IOR task to be as disruptive to object storage daemons as possible by having it create many small synchronous writes from as many processes as possible throughout the runtime of the compute tasks. Since the I/O tasks are designed to run through the entire compute task, we structured the IOR job so that it would not reasonably terminate during the computation. Once the computation was complete, the IOR task was killed. Table~\ref{tab:ior-params} describes the specific options we used for IOR.

\begin{table}[htbp]
\caption{IOR Parameters}
\begin{center}
\begin{tabular}{|c|c|c|}
  \hline
  \textbf{Parameter} & \textbf{Description} & \textbf{Value} \\
%\textbf{Node Count}&\textbf{Row Count (N)} & \textbf{Grid P} & \textbf{Grid Q} \\
  \hline
      [srun] -n & Processes (per node) & 56 \\
      -t & Transfer size (bytes) & 512 \\
      -T & Maximum run duration (minutes) & 20 \\
      -D & Stonewalling deadline (seconds) & 60 \\
      -i & Test repetitions & 1048576 \\
      -e & Sync after each write phase & enabled \\
      -C & Reorder tasks & enabled \\
      -w & Perform write test & enabled \\
      -a & Access method & POSIX \\
      -s & Number of segments & 1024 \\
      -F & Use file-per-process & enabled \\
      -Y & Sync after every write & enabled \\      
\hline
\end{tabular}
\label{tab:ior-params}
\end{center}
\end{table}




 %Methods subsection
\section{Results}

\begin{figure*}[htbp]
\centerline{\includegraphics{multinode-hpl-runtime-impact}}
\caption{Execution times of HPL with and without IOR processes co-located within the partition. Error bars indicate 95\% confidence interval.}
\label{fig:multinode}
\end{figure*}

This set of experiment included a small set of anomalies that we note here for completeness, but do not affect our analysis. All runs were completed between 7 and 10 times, with few exceptions. First, since we anticipated the ``Matching Lustre'' cases to be less variable, we only ran those experiments only three times each. Second, we were unable to get the largest Lustre run category (128-node HPL + 128-node IOR) completed before the deadline, as the system is a busy production platform. We will include this last data point before final publication. In the case of the 128-node HPL+BeeOND executions, we occasionally experienced runtimes slightly longer than the 20-minute IOR limit. We believe this did not significantly impact measurements, as the average runtime was less than 5\% longer than 20 minutes.

\begin{figure}[htbp]
\centerline{\includegraphics{multinode-95ci-lustre-beeond}}
\caption{A detailed view of HPL execution variance between HPL-only tasks (with BeeOND processes running) and HPL running alongside IOR targetting Lustre (without BeeOND processes running).}
\label{multinode-variance}
\end{figure}

{\bf Overall trends.} Based on our full results, shown in Figure~\ref{fig:multinode}, it is clear that introducing a storage workload on BeeOND daemons running alongside HPL cause statistically significant impact to HPL runtime. Introducing a single IOR process affected jobs of all sizes, causing the HPL-only job to increase its runtime by 7-13\% for 128 processes. Further increasing the I/O load with matching processes caused even greater execution times, with the 128-node Matching BeeOND (No Metadata) case experiencing 47-52\% extended runtime.

{\bf Metadata server effects.} In the ``Beeond Matching, Skip Metadata'' experiments, we were sure to keep the multi-node HPL from running on the node that hosted the metadata server. This was in contrast to the ``BeeOND Matching,'' where the multi-node HPL would overlap with the metadata server. The intention was to separate additional metadata tasks caused by our IOR workload to be from measurement, showing whether object storage processes caused less runtime impact generally. Our experiments did not definitively demonstrate a difference runtime. Determining the extent of any impact will require further experimentation focusing on metadata workloads, as well as how different types of workloads may be balanced by BeeOND.

{\bf Overhead of idle BeeOND daemons.} Figure~\ref{multinode-variance} shows a detailed view of a particularly interesting phenomenon. When running HPL-only configurations, we used the same job scripts that were used for BeeOND-enabled IOR tests. Therefore, the same Slurm constraint that caused BeeOND startup were included, so those processes were running on those nodes. However, for the Lustre-specific runs, we did not specify the BeeOND constraint, so did not load any BeeOND daemons. Despite a total absence of storage operations during the HPL-only jobs, the HPL-only jobs were slower than the Lustre+IOR jobs by a statistically significant margin. For the 64-node HPL cases, this impact was likely between 0.9 and 2.5\%. This impact grows with the size of the job, indicating that large scale runs could see even larger slowdowns. This is congruent with findings from previous work that showed daemon processes on Linux clusters could consume enough cycles to impact larger scale tasks~\cite{daemon-interference}. This was a surprising finding, and we intend to pursue this line of investigation further. While the confounding factor of the Lustre IOR prevents us from definitively asserting that idle BeeOND daemons caused an overhead, a simple variation of this experiment will definitively show whether this link exists. Such an experiment will be run and reported for an accepted version of this paper.

\section{Discussion}

In the era of massively multicore computing, where it is commonplace to see node architectures with dozens of cores, colocating system or user services with compute tasks is usually seen as a minor imposition on a running application. However, it is clear that using typical scheduling tactics and process layouts can significantly impact the runtime of applications. This has been demonstrated to be especially true at scale. How can we mitigate these impacts while continuing to offer useful services to HPC users?

Any solution compute interference should understand the requirements of the application and user expectation. While this work looks at the problem only from the perspective of compute impact, there is the related concern of I/O impact. Any compute impact mitigation has the potential to impact storage performance, which may be the more important to some users. Therefore, we encourage multiple, possibly conflicting mitigations to be made available to allow maximum flexibility for the end user.

One long-used mitigation strategy is core specialization, where some cores are dedicated to specific system services. This is becoming more pertinent with the advent of efficiency cores in modern CPUs. The main idea is that computations are pinned to some number of cores that are dedicated to their use, while other services are scheduled onto a separate subset of cores, allowing for less interference between them. One downside of this strategy is its static nature. A job submission is typically specified as the number of processes per node, so the split between specialized cores and service cores must be decided in advance. Over-estimating or under-estimating the number of cores required for services will constrain the available performance of the system. However, using so-called ``efficiency cores'' for services may mitigate this somewhat, as these cores are more tuned to handing tasks like I/O, creating a natural place to draw the line between compute and services. This can work well if overheads are expected to consume at least a full core, and compute task performance is important.

In absence of core specialization, CPU and network quotas can provide another means for limiting the impact of storage on compute. Simply indicating to the operating system that storage daemons may only consume some percentage of resources would control the maximum impact of these services. Relatedly, creating a mechanism to throttle network traffic from file system clients at the source will go some distance toward making the storage system self-regulating. Unfortunately, this would do little to mitigate hot spots, where all clients are attempting to access the same small set of servers for storage services. For tightly coupled, balanced jobs, that could cause broad performance degradation.

Another strategy for ensuring minimal application impact is to allow users to control where file system processes are located within a job. In the case of the example benchmark in this paper, simply starting all BeeOND processes on nodes that were engaged in IOR runs would exempt HPL-running nodes from any impacts. A downside for typical HPC architectures with direct-attached storage is the loss of use of SSDs kept within those nodes, causing underutilization and smaller/slower file systems. While it may be possible to use network protocols such as NVMe Over Fabrics on exempt nodes to share the block storage with the nodes running daemons, the host node may still experience some runtime impacts. Less risky but more costly solutions include architectures with disaggregated storage, designed as Just-a-Bunch-Of-Flash chassis. One can also simply expand the size of the job to include extra servers specifically for file system services, if the user anticipates high utilization.

\section{Conclusions and Future Work}

In this paper, we reviewed a set of experiments where we combined IOR, a storage benchmark, against HPL, a compute benchmark simulating a tightly coupled linear algebra application. We ran these benchmarks on a system with per-allocation BeeOND file system capabilities, which gives a user a private file system by running storage services on compute nodes alongside running applications. Our goal was to show whether heavy storage use could impact compute task performance.

We were able to show that, even for large compute jobs (128 nodes, 7168 cores) paired with a small number of I/O processes (1 node, 56 processes), there was significant impact to the HPL runtimes. In some cases, HPL runtime was increased by up to 52\%. Surprisingly, we were also able to measure a potential impact (up to 2.5\% on 64 nodes) from simply running BeeOND processes alongside HPL without any I/O traffic. Such a phenomenon indicates that running BeeOND daemons on dedicated cores is a reasonable strategy as job node counts increase.

This research has uncovered a large number of future research directions. While we have uncovered CPU-based performance impacts from I/O daemons, we did not receive clear results regarding impact of metadata services. This is likely because the IOR benchmark parameters we used did not heavily exercise the metadata facilities of BeeOND. We intend to design new experiments that will specifically exercise metadata operations. We also intend to explore user-controlled placement of BeeOND processes to mitigate compute performance impact.


\bibliographystyle{plain}
\bibliography{ieeecluster}

\end{document}
